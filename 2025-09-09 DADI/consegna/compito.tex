\documentclass[12pt]{article}
\usepackage[pdftex]{graphicx}
\usepackage{amsfonts}
\usepackage[italian]{babel}
\usepackage{graphicx}
\usepackage{color}
\usepackage{multirow,bigdelim}
\usepackage{relsize}
\usepackage{fdsymbol}
\usepackage{mdframed}

\definecolor{grey}{rgb}{0.3,0.3,0.3}
\definecolor{verylightgray}{rgb}{.97,.97,.97}
\definecolor{lightred}{rgb}{1,.70,.70}

\usepackage{listings, framed}
\lstset{
  language=Java,
  showstringspaces=false,
  columns=flexible,
  basicstyle={\small\ttfamily},
  frame=none,
  numbers=none,
  keywordstyle=\bfseries\color{grey},
  commentstyle=\itshape\color{red},
  identifierstyle=\color{black},
  stringstyle=\color{blue},
  numberstyle={\ttfamily},
%  breaklines=true,
  breakatwhitespace=true,
  tabsize=3,
  escapechar=|
}

\mdfsetup{font=\scriptsize}

\def\codesize{\smaller}
\def\<#1>{\codeid{#1}}
\newcommand{\codeid}[1]{\ifmmode{\mbox{\codesize\ttfamily{#1}}}\else{\codesize\ttfamily #1}\fi}

%****************enlarge layout
\textheight     243.5mm
\topmargin      -20.0mm
\textwidth      500pt
\hoffset        -80pt
%*****************theorems and such
\newcounter{esnu}
\newenvironment{esercizio}{\medskip \noindent {\bf Esercizio\addtocounter{esnu}{1} \arabic{esnu}}}{}
\pagestyle{empty}
\newcommand{\liff}{\mathrel{\leftrightarrow}}   % Logical IFF Symbol
\newcommand{\metaiff}{\Longleftrightarrow}      %iff in metatheory

\begin{document}

\begin{center}
  \textbf{Esame di Programmazione II, 9 settembre 2025}\\
  (si consegnino i file \<.java> delle classi richieste negli esercizi)
\end{center}

\emph{
Si crei un progetto Eclipse e
il package \<it.univr.dadi>. Si copino al suo interno
le classi del compito.
Non si modifichino le dichiarazioni dei metodi e delle classi. Si possono definire altri campi,
metodi o costruttori non richiesti dal compito, ma devono essere \<private>. Si possono definire altre classi
non richieste dal compio, che in tal caso vanno consegnate.
La soluzione che verr\`a consegnata dovr\`a compilare,
altrimenti non verr\`a corretta.}

\vspace*{2ex}

L'interfaccia \<Dado>, gi\`a completa e da non modificare,
rappresenta un dado, cio\`e un oggetto che ha un numero fissato di facce
e che pu\`o essere \emph{lanciato}, ottenendo un numero intero fra $1$ e il numero di facce
del dado (inclusi). La classe \<Lanci> implementa il lancio ripetuto
di alcuni dadi, forniti al costruttore. I suoi oggetti possono essere stampati, poich\'e il
\<toString()> restituisce il risultato dei lanci, e possono essere trasformati in una stringa
che disegna un istogramma con il risultato dei lanci, tramite il metodo \<frequenze()>.

Per esempio, si consideri il seguente codice della classe \<Main>, gi\`a fatta e completa:

\begin{lstlisting}[language=Java]
System.out.println("Lanciamo 20 volte due dadi a sei facce");
Lanci l = new Lanci(20, new D6(), new D6());
System.out.println("Lanci ottenuti: " + l);
System.out.println(l.frequenze());

System.out.println("Lanciamo 10000 volte un dado a sei facce");
l = new Lanci(10000, new D6());
System.out.println(l.frequenze());

System.out.println("Lanciamo 10000 volte un dado a sei facce truccato");
l = new Lanci(10000, new D6Truccato());
System.out.println(l.frequenze());

System.out.println("Lanciamo 10000 volte un dado a sei facce truccatissimo");
l = new Lanci(10000, new D6Truccatissimo());
System.out.println(l.frequenze());

System.out.println("Lanciamo 10000 volte un dado a otto facce, uno a sei facce truccato
  e uno a dieci facce");
l = new Lanci(10000, new D8(), new D6Truccato(), new D10());
System.out.println(l.frequenze());

System.out.println("Lanciamo 10000 volte tre dadi a sei facce, usando frecce");
l = new LanciFrecce(10000, new D6(), new D6(), new D6());
System.out.println(l.frequenze());
		
System.out.println("Lanciamo 10000 volte tre dadi a sei facce, usando frecce alternate");
l = new LanciFrecceAlternate(10000, new D6(), new D6(), new D6());
System.out.println(l.frequenze());

System.out.println("Lanciamo -10000 volte tre dadi a sei facce, usando frecce alternate");
new LanciFrecceAlternate(-10000, new D6(), new D6(), new D6());
\end{lstlisting}

\noindent
L'esecuzione di tale codice dovrebbe stampare qualcosa del tipo:
%
\begin{mdframed}[backgroundcolor=lightred]
  {\small\begin{verbatim}
Lanciamo 20 volte due dadi a sei facce
Lanci ottenuti: [8, 8, 4, 8, 10, 8, 4, 7, 8, 5, 5, 6, 5, 6, 8, 6, 6, 8, 6, 8]
  2:  (0.0%)
  3:  (0.0%)
  4: ********** (10.0%)
  5: *************** (15.0%)
  6: ************************* (25.0%)
  7: ***** (5.0%)
  8: **************************************** (40.0%)
  9:  (0.0%)
 10: ***** (5.0%)
 11:  (0.0%)
 12:  (0.0%)

Lanciamo 10000 volte un dado a sei facce
  1: **************** (16.7%)
  2: **************** (16.3%)
  3: **************** (16.5%)
  4: ***************** (17.3%)
  5: **************** (16.1%)
  6: ***************** (17.0%)

Lanciamo 10000 volte un dado a sei facce truccato
  1: ********** (10.5%)
  2: ********** (10.3%)
  3: ********* (10.0%)
  4: ********** (10.2%)
  5: ***************************** (29.5%)
  6: ***************************** (29.4%)

Lanciamo 10000 volte un dado a sei facce truccatissimo
  1:  (0.0%)
  2:  (0.0%)
  3:  (0.0%)
  4:  (0.0%)
  5:  (0.0%)
  6: *******************************......******************* (100.0%)

Lanciamo 10000 volte un dado a otto facce, uno a sei facce truccato
    e uno a dieci facce
  3:  (0.2%)
  4:  (0.5%)
  5:  (0.8%)
  6: * (1.3%)
  7: ** (2.3%)
  8: *** (3.5%)
  9: **** (4.9%)
 10: ***** (5.6%)
 11: ******* (7.1%)
 12: ******** (8.2%)
 13: ******** (8.3%)
 14: ********* (9.7%)
 15: ********* (9.8%)
 16: ******** (8.2%)
 17: ******* (7.8%)
 18: ****** (6.5%)
 19: ***** (5.3%)
 20: **** (4.1%)
 21: ** (2.9%)
 22: * (1.7%)
 23:  (0.9%)
 24:  (0.4%)

Lanciamo 10000 volte tre dadi a sei facce, usando frecce
  3:  (0.4%)
  4: > (1.5%)
  5: --> (3.1%)
  6: ---> (4.5%)
  7: ------> (7.2%)
  8: --------> (9.5%)
  9: ----------> (11.0%)
 10: -----------> (12.9%)
 11: -----------> (12.6%)
 12: ----------> (11.5%)
 13: --------> (9.9%)
 14: -----> (6.4%)
 15: ---> (4.5%)
 16: -> (2.8%)
 17: > (1.6%)
 18:  (0.4%)

Lanciamo 10000 volte tre dadi a sei facce, usando frecce alternate
  3:  (0.4%)
  4: > (1.3%)
  5: => (2.5%)
  6: ---> (4.8%)
  7: =====> (6.7%)
  8: --------> (9.9%)
  9: ==========> (11.2%)
 10: -----------> (13.0%)
 11: ===========> (12.6%)
 12: ----------> (11.7%)
 13: ========> (9.3%)
 14: -----> (6.8%)
 15: ===> (4.9%)
 16: -> (3.0%)
 17: > (1.3%)
 18:  (0.6%)
Lanciamo -10000 volte tre dadi a sei facce, usando frecce alternate:
java.lang.IllegalArgumentException: Il numero di lanci richiesto deve essere positivo
\end{verbatim}}
\end{mdframed}

\vspace*{2ex}\textbf{Esercizio 1 ($3$ punti).}
Si completi la classe astratta \<AbstractDado>, che implementa il codice comune a tutti i dadi,
cio\`e quello per la gestione del numero di facce del dado.

\vspace*{1.5ex}\textbf{Esercizio 2 ($3$ punti).}
Si scriva una sottoclasse astratta \<DadoUniforme> di \<AbstractDado>, che implementa un dado
uniforme, in cui cio\`e la probabilit\`a di ottenere una faccia,
lanciandolo, \`e la stessa per ciascuna faccia. Questa classe dovr\`a avere un costruttore
che specifica il numero di facce del dado e dovr\`a implementare il metodo \<lancio()>.

\vspace*{1.5ex}\textbf{Esercizio 3 ($3$ punti).}
Si scrivano le sottoclassi concrete \<D6>, \<D8> e \<D10> di \<DadoUniforme>, che implementato
dadi uniformi a sei, otto e dieci facce, rispettivamente.

\vspace*{1.5ex}\textbf{Esercizio 4 ($3$ punti).}
Si scriva una sottoclasse \<D6Truccato> di \<AbstractDado> che implementa un dado a sei facce
truccato, poich\'e, lanciandolo, le facce 5 e 6 escono ciascuna mediamente
nel 30\% dei casi, mentre le facce 1, 2, 3 e 4 escono ciascuna mediamente
nel 10\% dei casi.

\vspace*{1.5ex}\textbf{Esercizio 5 ($2$ punti).}
Si scriva una sottoclasse \<D6Truccatissimo> di \<AbstractDado> che implementa un dado a sei facce
truccatissimo, poich\'e, lanciandolo, esce sempre e soltanto la faccia 6.

\vspace*{1.5ex}\textbf{Esercizio 6 ($8$ punti).}
Si completi la classe concreta \<Lanci>, il cui costruttore deve lanciare i dadi forniti,
insieme e ripetutamente, per il numero di volte \<quanti> indicato al costruttore.
Il metodo \<toString()> deve restituire la stampa dei risultati di tali lanci.
Il metodo \<frequenze()> deve restituire una stringa istogramma, come quelli che si vedono nell'esecuzione
dell'esempio
\<Main> della pagina precedente. La classe \<Lanci> usa asterischi per rappresentare le barre degli
istogrammi.

\vspace*{1.5ex}\textbf{Esercizio 7 ($4$ punti).}
Si completi la sottoclasse concreta \<LanciFrecce> di \<Lanci>, che si differenzia da \<Lanci>
solo perch\'e rappresenta le barre degli istogrammi con delle frecce, come nel penultimo esempio
della pagina precedente.

\vspace*{1.5ex}\textbf{Esercizio 8 ($5$ punti).}
Si completi la sottoclasse concreta \<LanciFrecceAlternate> di \<Lanci>, che si differenzia da \<Lanci>
solo perch\'e rappresenta le barre degli istogrammi con delle frecce, alternativamente fatte con il
carattere \texttt{'-'} e con il carattere \texttt{'='}, come nell'ultimo esempio
della pagina precedente.

\end{document}
